% Copyright (C) 2014-2016 by Thomas Auzinger <thomas@auzinger.name>

\documentclass[draft,final]{vutinfth} % Remove option 'final' to obtain debug information.

% Load packages to allow in- and output of non-ASCII characters.
\usepackage{lmodern}        % Use an extension of the original Computer Modern font to minimize the use of bitmapped letters.
\usepackage[T1]{fontenc}    % Determines font encoding of the output. Font packages have to be included before this line.
\usepackage[utf8]{inputenc} % Determines encoding of the input. All input files have to use UTF8 encoding.

% Extended LaTeX functionality is enables by including packages with \usepackage{...}.
\usepackage{amsmath}    % Extended typesetting of mathematical expression.
\usepackage{amssymb}    % Provides a multitude of mathematical symbols.
\usepackage{mathtools}  % Further extensions of mathematical typesetting.
\usepackage{microtype}  % Small-scale typographic enhancements.
\usepackage[inline]{enumitem} % User control over the layout of lists (itemize, enumerate, description).
\usepackage{multirow}   % Allows table elements to span several rows.
\usepackage{booktabs}   % Improves the typesettings of tables.
\usepackage{subcaption} % Allows the use of subfigures and enables their referencing.
\usepackage[ruled,linesnumbered,algochapter]{algorithm2e} % Enables the writing of pseudo code.
\usepackage[usenames,dvipsnames,table]{xcolor} % Allows the definition and use of colors. This package has to be included before tikz.
\usepackage{nag}       % Issues warnings when best practices in writing LaTeX documents are violated.
\usepackage{todonotes} % Provides tooltip-like todo notes.
\usepackage{hyperref}  % Enables cross linking in the electronic document version. This package has to be included second to last.
\usepackage[acronym,toc]{glossaries} % Enables the generation of glossaries and lists fo acronyms. This package has to be included last.

% Define convenience functions to use the author name and the thesis title in the PDF document properties.
\newcommand{\authorname}{Rebeka Koszticsak} % The author name without titles.
\newcommand{\thesistitle}{Illustrative Thumbnails} % The title of the thesis. The English version should be used, if it exists.

% Set PDF document properties
\hypersetup{
    pdfpagelayout   = TwoPageRight,           % How the document is shown in PDF viewers (optional).
    linkbordercolor = {Melon},                % The color of the borders of boxes around crosslinks (optional).
    pdfauthor       = {\authorname},          % The author's name in the document properties (optional).
    pdftitle        = {\thesistitle},         % The document's title in the document properties (optional).
    pdfsubject      = {Subject},              % The document's subject in the document properties (optional).
    pdfkeywords     = {a, list, of, keywords} % The document's keywords in the document properties (optional).
}

\setpnumwidth{2.5em}        % Avoid overfull hboxes in the table of contents (see memoir manual).
\setsecnumdepth{subsection} % Enumerate subsections.

\nonzeroparskip             % Create space between paragraphs (optional).
\setlength{\parindent}{0pt} % Remove paragraph identation (optional).

\makeindex      % Use an optional index.
\makeglossaries % Use an optional glossary.
%\glstocfalse   % Remove the glossaries from the table of contents.

% Set persons with 4 arguments:
%  {title before name}{name}{title after name}{gender}
%  where both titles are optional (i.e. can be given as empty brackets {}).
\setauthor{}{\authorname}{}{female}
\setadvisor{Dr. techn.}{Manuela Waldner}{Msc.}{female}

% For bachelor and master theses:
%\setfirstassistant{Pretitle}{Forename Surname}{Posttitle}{male}
%\setsecondassistant{Pretitle}{Forename Surname}{Posttitle}{male}
%\setthirdassistant{Pretitle}{Forename Surname}{Posttitle}{male}

% For dissertations:
\setfirstreviewer{Pretitle}{Forename Surname}{Posttitle}{male}
\setsecondreviewer{Pretitle}{Forename Surname}{Posttitle}{male}

% For dissertations at the PhD School and optionally for dissertations:
\setsecondadvisor{Pretitle}{Forename Surname}{Posttitle}{male} % Comment to remove.

% Required data.
\setaddress{Address}
\setregnumber{1325492}
\setdate{01}{01}{2001} % Set date with 3 arguments: {day}{month}{year}.
\settitle{\thesistitle}{Illustrative Thumbnails} % Sets English and German version of the title (both can be English or German).
%\setsubtitle{Optional Subtitle of the Thesis}{Optionaler Untertitel der Arbeit} % Sets English and German version of the subtitle (both can be English or German).

% Select the thesis type: bachelor / master / doctor / phd-school.
% Bachelor:
\setthesis{bachelor}
%
% Master:
%\setthesis{master}
%\setmasterdegree{dipl.} % dipl. / rer.nat. / rer.soc.oec. / master
%
% Doctor:
%\setthesis{doctor}
%\setdoctordegree{rer.soc.oec.}% rer.nat. / techn. / rer.soc.oec.
%
% Doctor at the PhD School
%\setthesis{phd-school} % Deactivate non-English title pages (see below)

% For bachelor and master:
\setcurriculum{Media Informatics and Visual Computing}{Medieninformatik und Visual Computing} % Sets the English and German name of the curriculum.

% For dissertations at the PhD School:
\setfirstreviewerdata{Affiliation, Country}
\setsecondreviewerdata{Affiliation, Country}


\begin{document}

\frontmatter % Switches to roman numbering.
% The structure of the thesis has to conform to
%  http://www.informatik.tuwien.ac.at/dekanat

\addtitlepage{naustrian} % German title page (not for dissertations at the PhD School).
\addtitlepage{english} % English title page.
\addstatementpage

\begin{danksagung*}
\todo{Ihr Text hier.}
\end{danksagung*}

\begin{acknowledgements*}
\todo{Enter your text here.}
\end{acknowledgements*}

\begin{kurzfassung}
\todo{Ihr Text hier.}
\end{kurzfassung}

\begin{abstract}
Thumbnails are used to display the screens when switching between them on the computer
and on mobile devices. These images make it easier to recognize the opened applications,
and help to find the needed window quicker. Thumbnails display however only a screenshot
of the windows, so they get potentially confusing if there are more opened windows or if the
same application is opened multiple times. Depending on the resolution of the display, the
screenshot size decreases as the number of opened windows increases. Furthermore,
within the same application (like MS Office World) the screenshots are similar in appearance
(eg.: white paper and tool bar), but the important text is not readable.
There are several approaches that filter the important areas of the images to make editting
less obvious or enhance the main region. In this bachelor thesis an application
is implemented that uses these methods on the screencaptured
images. The less important
areas of the screenshots are cut off, and the thumbnails show only important information,
which makes them more illustrative and easier to fulfill their purpose.
\end{abstract}

% Select the language of the thesis, e.g., english or naustrian.
\selectlanguage{english}

% Add a table of contents (toc).
\tableofcontents % Starred version, i.e., \tableofcontents*, removes the self-entry.

% Switch to arabic numbering and start the enumeration of chapters in the table of content.
\mainmatter

\chapter{Introduction}
\todo{Enter your text here.}

\chapter{Related Work}
There are several image processing algorithms, which can be helpful by creating illustrative thumbnails.
The main difference between them is, if they consider the fact that the input images always are screenshots.
Therefore, this section is divided into two parts.
The first one discusses algorithms with UI processing segments.
Algorithms, invented for retrieve visual data from common images, are examinated in the second section. 
According to the information visualization method, like combining the most important parts in form of collages or simple resizing, two classes are divided.
In the following some of these methods are compared.

\section{Processing UI elements}
Considering that the input is a screenshot, there is a high chance that common UI elements are shown on the screen.
Exceptions are only the cases where graphics, images, videos or application containing these, e.g. video games, gallery program, video player are captured.
The applications like that tend to hide all UI elements, "fullscreen" mode, or redefine them, e.g. game menu.\par
Labeling the image parts as content and non-content, the metadata about the UI elements can be helpful.
Forras uses already existing accessibility APIs to segment UI and non-UI data. 
Matching the metadata with the screen content provides a fast and robust result about the location of any kind of UI content.
There are howerer several disadvantages, that need to be take into account when using such APIs.
The range and the granularity of the support is often not wide enough.
The use of an accessibilty API does not make sure, that every UI element will be recognized, because some metadata is not reachable or it will be ignored by the application.\par
Because of that Forras Prefab works with its own prototypes. 
In the database models and  prototypes of common UI elements are saved.
The (components of) UI elements has to be matched with the prototypes in the data base, and after that the predefined metadata can be accessed.
Since the Prefab system is able to split complex widgets into their base elements, the database does not need to be unnecessary big, but it is still able to cover the most common UI elements.
In the case of special or rare UI widgets, like in a video game application or elements of a not not widely used software, the system fails.
If a widget is not saved in the database, there is no way, that it will be recognized.\par
Forras Sikuli offers a solution not for the incompleteness of the database, but for the granularity issues too.
It uses its own templates just like the Prefab system, but only in case of small icons and widgets.
Since in case of larger objects template matching would be too expensive, after the processing of a training pattern, the Sikuli system is able to create new object models too.
Although this feature is used in the application for other purpose, i.e. reduce the matching cost, it can solve the problem of database and of the granularity.
Sikuli makes it possible to expand the database and to make the database entry afterwards more detailed.\par
But in case of accessibility APIs not only the availability of metadata causes possible problems.
It has also no information about the actual visibilty of the UI elements.
One window or rather one widget can hide an other one, some content can be out of the range of the borders of the screen etc.    
Forras is based on the Prefab system, but it builds a hierarchical tree from the widgets. 
The content is found at the leafs, and the parents are the widget, where the child is built in.
With the help of this tree the order of the UI elements gets clear, and so the misleading information can be eliminated.\par
After the labeling of the UI elements correctly, they can be processed as needed.
They can cut off as whole or processed according to the information content.
Forras invented an algorithm to eliminate objects of a picture and fill their place with the texture of the object behind them using the z-buffer information.
The tree mentioned above works as a z-buffer in this case.
With this cut off algorithm unnecessary widgets can be easily eliminated and more important elements of the UI can so get better visible.\par
According to the definition of "illustrative" of this thesis the UI elements of a screen has to get cut so the segmentation of the UI elements is not needed.
Although the Prefab and Sikuli systems can be helpful for labeling the images as UI and content.
But theses approaches has the drawback of the usage of the database and the slow template matching. 
Since there is no need to know, which widgets are on the screen, and processing the actual content can already cause performance issues, the use these methods would overcomplicate the application without providing noteworthy advantages.


  

\chapter{Methodology}



% Remove following line for the final thesis.
%\input{intro.tex} % A short introduction to LaTeX.

\backmatter

% Use an optional list of figures.
\listoffigures % Starred version, i.e., \listoffigures*, removes the toc entry.

% Use an optional list of tables.
\cleardoublepage % Start list of tables on the next empty right hand page.
\listoftables % Starred version, i.e., \listoftables*, removes the toc entry.

% Use an optional list of alogrithms.
\listofalgorithms
\addcontentsline{toc}{chapter}{List of Algorithms}

% Add an index.
\printindex

% Add a glossary.
\printglossaries

% Add a bibliography.
\bibliographystyle{alpha}
\bibliography{intro}

\end{document}